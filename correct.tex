\documentclass{llncs}
\usepackage[utf8]{inputenc}
\usepackage{graphicx}
\usepackage{amsmath}
\usepackage{amssymb}
\usepackage{amsfonts}

\newcommand{\ind}[1]{\textbf{1}_{\left \{ #1 \right \}}}
\newtheorem{propn}{Proposition}
\newtheorem{obs}{Observation}
\DeclareMathOperator*{\argmin}{\arg\!\min}

\title{Anti-Coordination Games and Stable \\Graph Colorings}

\begin{document}
\author{Jeremy Kun \and Brian Powers \and Lev Reyzin}

\institute{Department of Mathematics, Statistics, and Computer Science\\
University of Illinois at Chicago\\
\texttt{\{jkun2,bpower6,lreyzin\}@math.uic.edu}}

\maketitle

\begin{abstract}
Motivated by understanding non-strict and strict pure strategy equilibria in
network anti-coordination games, we define notions of stable and, respectively,
strictly stable colorings in graphs.  We characterize the cases when such
colorings exist and when the decision problem is NP-hard. These correspond to
finding pure strategy equilibria in the anti-coordination games, whose price of
anarchy we also analyze.  We further consider the directed case, a
generalization that captures both coordination and anti-coordination. We prove
the decision problem for non-strict equilibria in directed graphs is NP-hard.
Our notions also have multiple connections to other combinatorial questions,
and our results resolve some open problems in these areas, most notably the
complexity of the strictly unfriendly partition problem.
\end{abstract}

\section{Introduction}

Anti-coordination games form some of the basic payoff structures in game
theory.  Such games are ubiquitous; miners deciding which land to drill for
resources, company employees trying to learn diverse skills, and airplanes
selecting flight paths all need to mutually anti-coordinate their strategies in
order to maximize their profits or even avoid catastrophe.

\end{document}
